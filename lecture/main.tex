\documentclass[]{article}

\usepackage{hyperref}
\usepackage{import}
\usepackage{listings}
\usepackage{minted}
\usepackage[dvipsnames]{xcolor}
\usepackage{hyperref}
\usepackage{graphicx}
\usepackage{geometry}
 \geometry{
 a4paper,
 left=2.2cm,
 right=2.2cm,
 top=2.5cm,
 bottom=2.5cm
 }

\usepackage[localise]{xepersian}
\settextfont{Vazirmatn}
\setlatintextfont{Roboto}

% for listing environments
\colorlet{LightGray}{Gray!30!}

% for href
\eqcommand{تارنما}{href}

\عنوان{برنامه‌نویسی سوکت}
\نویسنده{پرهام الوانی}
\تاریخ{پاییز ۱۳۹۸}

\begin{document}
  \عنوان‌ساز
  \فهرست‌مطالب
  \صفحه‌شکن

  \قسمت{مقدمه}
  در این آموزش قصد داریم یک سرور و کلاینت‌ با استفاده از زبان پایتون پیاده‌سازی کنیم.
  در این برنامه کلاینت‌ پس از متصل شدن به سرور، متنی را به سرور ارسال می‌کند.
  سرور پس از دریافت متن، آن را دوباره به کلاینت‌ ارسال می‌کند. کلاینت‌ با دریافت‌ دوباره متن، متن دریافت شده را چاپ کرده و اتصال را قطع می‌کند.
  به چنین ابزارهایی \متن‌لاتین{Echo} می‌گویند که در گذشته از آن‌ها برای صحت‌سنجی ارتباط استفاده می‌شد.

  \قسمت{سرور \متن‌لاتین{Echo}}

  کد سرور به شرح زیر است، هر بخش کد به تفکیک در ادامه توضیح داده می‌شود.

  \begin{latin}
  \begin{minted}[bgcolor=LightGray]{python}
import socket

HOST = '127.0.0.1'  # Standard loopback interface address (localhost)
PORT = 1373         # Port to listen on (non-privileged ports are > 1023)

with socket.socket(socket.AF_INET, socket.SOCK_STREAM) as s:
    s.bind((HOST, PORT))
    s.listen()
    conn, addr = s.accept()
    with conn:
        print('Connected by', addr)
        while True:
            data = conn.recv(1024)
            if not data:
                break
            conn.sendall(data)
  \end{minted}
  \end{latin}

 در ابتدا یک سوکت‌ ایجاد می‌کنیم. کد زیر برای ایجاد سوکت‌ استفاده می‌شود:

  \begin{latin}
  \begin{minted}[bgcolor=LightGray]{python}
socket.socket(socket.AF_INET, socket.SOCK_STREAM)
  \end{minted}
  \end{latin}

  ورودی اول تابع \متن‌لاتین{socket} نوع آدرس‌دهی و ورودی دوم نوع انتقال داده است‌.
  در این مثال با انتخاب \متن‌لاتین{AF\_INET} نوع آدرس‌دهی را از نوع \متن‌لاتین{IPv4} انتخاب می‌کنیم‌.
  در صورتی که قصد داریم آدرس را به صورت \متن‌لاتین{IP} نوع چهارم یا \متن‌لاتین{hostname} استفاده کنیم
  باید این نوع از آدرس‌دهی را به عنوان ورودی به تابع بدهیم‌.
  \متن‌لاتین{SOCK\_STREAM} برای ارتباط بر روی پروتکل \متن‌لاتین{TCP} می‌باشد.
  در صورتی که قصد داشته باشیم پروتکل لایه انتقال را تغییر دهیم از این ورودی استفاده می‌کنیم.

  از انواع آدرس‌دهی دیگر می‌توان به \متن‌لاتین{AF\_PACKET}، \متن‌لاتین{AF\_INET6}، \متن‌لاتین{AF\_CAN}، \متن‌لاتین{AF\_BLUETOOTH} و \نقاط‌خ اشاره کرد.

  پس از ایجاد سوکت‌ آدرس مربوطه را به آن تخصیص می‌دهیم.
  در اینجا آدرس داده شده به سوکت‌ آدرسی‌ است که سوکت اطلاعات را از آن دریافت و از طریق آن ارسال می‌کند.
  آدرس شامل دوبخش \متن‌لاتین{host} و \متن‌لاتین{port} است.
  \متن‌لاتین{host} یک رشته است که مشخص می‌کند این عمل گوش دادن روی کدام کارت شبکه صورت بگیرد و قسمت \متن‌لاتین{port} مشخص کننده‌ی پورتی است که سرور روی آن گوش می‌دهد.
  مقدار \متن‌لاتین{host} با توجه به نوع آدرس مشخص شده می‌تواند به صورت \متن‌لاتین{IP} نوع چهارم،  \متن‌لاتین{hostname} یا رشته خالی باشد.
  در صورتی که رشته خالی یا \متن‌ایتالیک{\متن‌لاتین{``0.0.0.0''}} به عنوان \متن‌لاتین{host} تنظیم شود،
  سرور از تمام کارت های شبکه خود برای ارتباط گیری استفاده می‌کند.
  در نهایت با مشخص شدن این اطلاعات می‌خواهیم که سوکت در حالت قبول تقاضا قرار بگیرد.

  پارامتر ورودی تابع \متن‌لاتین{listen} که \متن‌لاتین{backlog} نام دارد تعداد ارتباط‌های پذیرفته نشده‌ای را مشخص می‌کند که سیستم عامل
  پیش از نپذیرفتن ارتباط‌های جدید خواهد پذیرفت. این تعداد به برنامه شما اجازه می‌دهد تا پذیرفتن درخواست‌ها را به دلایل مختلف به تعویق بیاندازد.

  \begin{latin}
  \begin{minted}[bgcolor=LightGray]{python}
s.bind((HOST, PORT))
s.listen(1)
  \end{minted}
  \end{latin}

  پس از مشخص شدن اطلاعات آدرس، سوکت باید منتظر برقرای ارتباط بماند.
  برنامه پس از برخورد با کد زیر در این همان جا متوقف می‌شود و منتظر دریافت درخواست ایجاد یک ارتباط توسط کلاینت می‌ماند.
  پس از دریافت درخواست برقراری ارتباط، سرور با فراخوانی کد زیر درخواست را تایید می‌کند.
  پس از تایید درخواست، آدرس درخواست دهنده و ارتباط ایجاد شده در دو متغیر \متن‌لاتین{addr} و \متن‌لاتین{conn} ذخیره می‌گردد.

  \begin{latin}
  \begin{minted}[bgcolor=LightGray]{python}
conn, addr = s.accept()
  \end{minted}
  \end{latin}

  حالا که ارتباط برقرار شده است، امکان انتقال اطلاعات فراهم می‌باشد.
  سرور هر بار به اندازه مشخصی از داده را می‌خواند. در این نمونه کد در هر بار \متن‌سیاه{حداکثر} مقدار ۱۰۲۴ بایت داده خوانده می‌شود.
  در عمل شما زمانی که به طراحی پروتکل مشغول هستید باید روشی برای مشخص کردن هر درخواست داشته باشید، مثلا می‌توانید هر درخواست
  را با کاراکتر \متن‌ایتالیک{\متن‌لاتین{``\\n''}} خاتمه دهید در این صورت می‌بایست آرایه دریافتی را برای پیدا کردن این کاراکتر جستجو کنید،
  در صورتی که این کاراکتر در آرایه شما موجود بود باقی آرایه را برای مرحله بعدی ذخیره کنید یا ممکن است که این کاراکتر را پیدا نکنید که
  در این صورت می‌بایست تا خواندن دوباره صبر کنید. در نظر داشته باشید که کتابخانه‌های سطح بالاتری برای ساده کردن این عملیات وجود دارند.

  \begin{latin}
  \begin{minted}[bgcolor=LightGray]{python}
with conn:
    print('Connected by', addr)
    while True:
        data = conn.recv(1024)
        if not data:
            break
  \end{minted}
  \end{latin}

  در اینجا ما از پروتکل خاصی پیروی نمی‌کنیم بنابراین پس از خواندن داده آن را دوباره به کلاینت باز می‌گردانیم.

  \begin{latin}
  \begin{minted}[bgcolor=LightGray]{python}
conn.sendall(data)
  \end{minted}
  \end{latin}

  \قسمت{کلاینت \متن‌لاتین{Echo} به وسیله‌ی \متن‌لاتین{Telnet}}

  پس از ایجاد سرور نیاز است که یک کلاینت نیز ایجاد کنیم تا بتوانیم با سرور ارتباط برقرار کنیم.
  یکی از راه‌های ساده تست سرور پیش از پیاده‌سازی کلاینت استفاده از دستور \متن‌لاتین{telnet} در سیستم عامل لینوکس می‌باشد.
  (این دستور در سایر سیستم‌عامل‌ها نیز وجود دارد.)
  ساختار ساده شده این دستور به شکل زیر است:

  \begin{latin}
  \begin{minted}[bgcolor=LightGray]{bash}
telnet [ip] [port]
  \end{minted}
  \end{latin}

  دستور \متن‌لاتین{telnet} یک ارتباط \متن‌لاتین{TCP} با آدرس \متن‌لاتین{IP} و پورت داده شده برقرار می‌کند.
  شما می‌توانید هر آنچه که می‌خواهید در این ارتباط نوشته و با فشردن دکمه‌ی \متن‌لاتین{enter} آن را به مقصدتان ارسال کنید.
  در ادامه اطلاعات دریافتی از سرور نیز برای شما به نمایش درخواهند آمد.

  دقت داشته باشید که سرور شما تنها یکبار عملیات \متن‌لاتین{accept} را انجام می‌دهد بنابراین بعد از بسته شدن ارتباط اولین کلاینت شما
  سرور شما به اتمام می‌رسد. برای جلوگیری از این مورد فراخوانی تابع \متن‌لاتین{accept} عموما در یک حلقه بی‌نهایت قرار می‌گیرد.

  \begin{latin}
  \begin{minted}[bgcolor=LightGray]{python}
import socket

HOST = '127.0.0.1'  # Standard loopback interface address (localhost)
PORT = 1373         # Port to listen on (non-privileged ports are > 1023)

with socket.socket(socket.AF_INET, socket.SOCK_STREAM) as s:
    s.bind((HOST, PORT))
    s.listen()
    while True:
      conn, addr = s.accept()
      with conn:
          print('Connected by', addr)
          while True:
              data = conn.recv(1024)
              if not data:
                  break
              conn.sendall(data)
  \end{minted}
  \end{latin}

  در این صورت سرور شما هرگز بسته نخواهد شد و شما می‌توانید به دفعات به آن متصل شوید.

  \قسمت{کلاینت Echo}

  \begin{latin}
  \begin{minted}[bgcolor=LightGray]{python}
import socket

HOST = '127.0.0.1'  # The server's hostname or IP address
PORT = 1373        # The port used by the server

with socket.socket(socket.AF_INET, socket.SOCK_STREAM) as s:
    s.connect((HOST, PORT))
    s.sendall('سلام دنیا'.encode())
    data = s.recv(1024)
    message = data.decode()
    print(f'received {message!r}')
  \end{minted}
  \end{latin}

مانند سرور در این بخش نیز نیاز است ابتدا یک سوکت با ویژگی های یکسان با سوکت ساخته شده در سرور بسازیم:

  \begin{latin}
  \begin{minted}[bgcolor=LightGray]{python}
socket.socket(socket.AF_INET, socket.SOCK_STREAM)
  \end{minted}
  \end{latin}

  پس از ایجاد سوکت، آدرس سرور را به سوکت داده تا درخواست ارتباط به سرور ارسال گردد.
  کد در این خط باقی می ماند تا زمانی که درخواست ارتباط توسط سرور قبول، رد و یا منقضی گردد.
  در صورت قبول نشدن درخواست برنامه خطا داده و متوقف می‌گردد.
  حتما در نظر داشته باشید که در هنگام استفاده از این سوکت در نرم افزارها، خطاهای بوجود آمده به علت عدم برقراری ارتباط به برنامه اصلی شما صدمه ای نزند.
  در صورت تایید ارتباط، برنامه ادامه می‌یابد.

  \begin{latin}
  \begin{minted}[bgcolor=LightGray]{python}
s.connect((HOST, PORT))
  \end{minted}
  \end{latin}

برای رسیدگی به خطاها در نظر داشته باشید که می‌توان از ساختار \متن‌لاتین{Try/Except} در پایتون استفاده کرد:

  \begin{latin}
  \begin{minted}[bgcolor=LightGray]{python}
try:
    with open('file.log') as file:
        read_data = file.read()
except:
    print('Could not open file.log')
  \end{minted}
  \end{latin}

  \begin{latin}
  \begin{minted}[bgcolor=LightGray]{python}
try:
    with open('file.log') as file:
        read_data = file.read()
except FileNotFoundError as fnf_error:
    print(fnf_error)
  \end{minted}
  \end{latin}

  کد زیر رشته \متن‌لاتین{Hello World} را به سرور ارسال می‌کند. در نظر داشته باشید که استفاده از تابع \متن‌لاتین{encode}
  به شما اجازه می‌دهد که رشته‌های \متن‌لاتین{UTF-8} را به رشته‌ای از بایت‌ها تبدیل کرده و ارسال کنید.

  \begin{latin}
  \begin{minted}[bgcolor=LightGray]{python}
s.sendall('Hello, world'.encode())
  \end{minted}
  \end{latin}

  پس از ارسال داده مانند قسمت سرور منتظر دریافت داده‌ها می‌مانیم. دقت داشته باشید که در عمل یکی از طرفیت می‌بایست ارتباط را خاتمه دهد.
  مثلا می‌توانیم در اینجا در سمت کلاینت پس از مطمئن شدم از دریافت صحیح رشته ارتباط را ببنیدم یا مثلا کاربر با دستور \متن‌لاتین{Ctrl + C}
  کلاینت و ارتباط را ببندد.

  \begin{latin}
  \begin{minted}[bgcolor=LightGray]{python}
while True:
  data = conn.recv(1024)
  if not data:
      break
  \end{minted}
  \end{latin}

  \قسمت{ارتباط مبتنی بر \متن‌لاتین{UDP}}
  زبان پایتون این امکان را می‌دهد که بتوانید با استفاده از پروتکل \متن‌لاتین{UDP} به پراکنده کردن داده بپردازید.
  در مثال پیشرو یک کلاینت \متن‌لاتین{UDP} برای ارسال پیام‌های همه‌پخشی ایجاد شده است.
  این کلاینت به آدرس \متن‌لاتین{0.0.0.0} و پورت \متن‌لاتین{44444} بسته شده و از آن برای دریافت بسته‌های \متن‌لاتین{UDP} استفاده می‌کند.
  این کلاینت بسته‌های را به آدرس همه‌پخشی و پورت \متن‌لاتین{37020} ارسال می‌کند.
  در نظر داشته باشید که در هردو این کلاینت‌ها ما تنظیمات مربوط به استفاده از وضعیت همه‌پخشی را انجام داده‌ایم.

  \begin{latin}
  \begin{minted}[bgcolor=LightGray]{python}
import socket
import time

server = socket.socket(socket.AF_INET, socket.SOCK_DGRAM)
server.setsockopt(socket.SOL_SOCKET, socket.SO_BROADCAST, 1)
server.settimeout(0.2)
server.bind(("", 44444))
message = b"your very important message"
while True:
    server.sendto(message, ('255.255.255.255', 37020))
    print("message sent!")
    time.sleep(1)
  \end{minted}
  \end{latin}

  کلاینت دوم در ادامه به آدرس \متن‌لاتین{0.0.0.0} و پورت \متن‌لاتین{37020} بسته شده و داده‌ها را دریافت می‌کند.
  در اینجا هم مثل ارتباط \متن‌لاتین{TCP} می‌بایست حداکثر داده‌ی دریافتی را مشخص کنیم ولی در نظر داشته باشید که اگر بسته‌ی دریافتی
  داده‌ی بیشتری از مقدار مشخص شده‌ی شما داشته باشد این داده‌ی اضافه \متن‌سیاه{از بین خواهد رفت}.

  \begin{latin}
  \begin{minted}[bgcolor=LightGray]{python}
import socket

client = socket.socket(socket.AF_INET, socket.SOCK_DGRAM) # UDP
client.setsockopt(socket.SOL_SOCKET, socket.SO_BROADCAST, 1)
client.bind(("", 37020))
while True:
    data, addr = client.recvfrom(1024)
    print("received message:",data,addr)
  \end{minted}
  \end{latin}

  تفاوت اصلی میان ارتباط \متن‌لاتین{TCP} و \متن‌لاتین{UDP} در چگونگی ساخت سوکت و توابع ارسال و دریافت اطلاعات است.
  البته از شبکه به خاطر دارید که در ارتباط \متن‌لاتین{UDP} یک اتصال شکل نمی‌گیرد
  و داده‌ها در قالب پیام به دست برنامه‌ی کاربردی می‌رسند.
  این پیام‌ها می‌توانند از ترتیب اولیه خود خارج شده باشند یا اینکه دچار خطا باشند.

  \فضای‌و*{\پر}
  \شروع{وسط‌چین}
  نسخه‌ی اولیه این گزارش در بهار ۱۳۹۸ توسط سپهر صبور حاضر شده است. نسخه‌ی حاضر تنها شامل بهبودهایی اندک نسبت به نسخه‌ی اصلی می‌باشد.
  این سند برپایه بسته \متن‌لاتین{\زی‌پرشین} گونه \متن‌لاتین{\گونه‌زی‌پرشین} توسعه پیدا کرده است.
  \پایان{وسط‌چین}

\end{document}
